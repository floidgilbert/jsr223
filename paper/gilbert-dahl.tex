% !TeX root = RJwrapper.tex
\title{\pkg{jsr223}: A Java Platform Integration for R with Programming Languages Groovy, JavaScript, JRuby, Jython, and Kotlin}
\author{by Floid R. Gilbert and David B. Dahl}

%///review bib
%///forgot to re-cite the programming languages. Look at old versions of papers.
%///change pkg to cranpkg where nec.
%///consider big int or big decimal calulations in statistical context for Groovy example...probably not.
%///review all syntax. I've changed some things including names of arguments. I've updated some code examples.
%///http://www.theserverside.com/feature/Top-7-takeaways-about-the-Java-platform-from-JavaOne-2015
%///update code sample original files. Made a lot of changes in-line.
%///use the language for state found here instead of using ``environment'': http://openjdk.java.net/jeps/222
%///update links to Oracle to openJDK instead.
%///can use other languages in the future, if they decide to do \pkg{jsr223} support. Scala has support...but it's not ready yet and David has a better package.
%///because of Ruby and Kotlin, every general example must be qualified accordingly when global variables are used. or, create a section that can be referred to in every case.
%///standardize functions with () or not. not. remove all ()
%///do a direct link to code examples? include readme.md file to explain.
\maketitle

\abstract{
The R package \CRANpkg{jsr223} is a high-level integration for the Java platform. It makes Java objects easy to use from within R; it provides a unified interface to integrate R with several programming languages; and it features extensive data exchange between R and Java. This integration leverages the performance benefits and cross-platform capabilities of the Java platform.

%The R package \pkg{jsr223} is a high-level integration for Java that makes Java objects easy to use from within R and simplifies bi-directional data exchange for a variety of data structures. Furthermore, \pkg{jsr223} employs the Java Scripting API to bring four scripting languages and, by extension, countless libraries to the R software environment. \pkg{jsr223} uses intuitive syntax to blend JavaScript, Ruby, Python, and Groovy (a Java-like scripting language) with R script. In all, this integration leverages the performance benefits and cross-platform capabilities of the Java Virtual Machine and extends the computing capabilities of the R software environment.
}

\hypertarget{introduction}{\section{Introduction}}

%About the same time Ross Ihaka and Robert Gentleman began developing R at the University of Auckland in the early 1990s, James Gosling and the so-called Green project team was working on a new programming language at Sun Microsystems in California. The Green team didn't set out to make a new language; rather, they were trying to move platform-independent, distributed computing into the consumer electronics marketplace. As Gosling explained, ``All along, the language was a tool, not the end.'' The Green project was short-lived. However, the resulting programming language, Java, flourished into one of the most popular development platforms in computing history. %Today, the Java computing environment is deployed in environments ranging from enterprise servers to mobile phones.

About the same time Ross Ihaka and Robert Gentleman began developing R at the University of Auckland in the early 1990s, James Gosling and the so-called Green project team was working on a new programming language at Sun Microsystems in California. The Green team didn't set out to make a new language; rather, they were trying to move platform-independent, distributed computing into the consumer electronics marketplace. As Gosling explained, ``All along, the language was a tool, not the end'' \citep{javainsidestory}. Unexpectedly, the programming language outlived the Green project and flourished into one of the most popular development platforms in computing history. That platform, Java, now powers applications ranging from the enterprise (\href{https://www.google.com/gmail/about/}{GMail}), to games (\href{https://minecraft.net}{Minecraft}), to interactive media (\href{https://en.wikipedia.org/wiki/Blu-ray}{Blu-ray}), to mobile devices (\href{https://www.android.com/}{Android}).

In 2003, Simon Urbanek released \pkg{rJava}, an integration package designed to avail R of the burgeoning development surrounding Java. The package has been very successful to this end. Today, it is one of the top-ranked solutions for R as measured by monthly downloads.\footnote{The \pkg{rJava} package ranks in the $95^{\text{th}}$ percentile for R package downloads according to \href{http://rdocumentation.org}{http://rdocumentation.org}.} \pkg{rJava} is described by Urbanek as a low-level R to Java interface analogous to \code{.C()} and \code{.Call()}, the built-in R functions for calling compiled C code. Like R's integration for C, \pkg{rJava} loads compiled code into an R process's memory space where it can be accessed via various R functions. Urbanek achieves this feat using the Java Native Interface (JNI), a standard framework that enables native (i.e. platform-dependent) code to access and use compiled Java code. The \pkg{rJava} API requires users to specify classes and data types in JNI syntax. One advantage to this approach is that the user has granular, direct access to Java classes. However, as with any low-level interface, the learning curve is relatively high and implementation requires verbose coding. Another advantage of using JNI is that it avoids the difficult task of dynamically interpreting or compiling source code. Of course, this is also a disadvantage: it limits \pkg{rJava} to using compiled code as opposed to embedding dynamic source code directly within R script.

Our \CRANpkg{jsr223} package builds on \pkg{rJava} to provide a high-level, extended interface to the Java platform. We use the Java Scripting API \citep{jsa} as defined by the specification “JSR-223: Scripting for the Java Platform” \citep{jsr223spec} to provide a unified integration interface to several programming languages. To date, \pkg{jsr223} supports five languages: Groovy, JavaScript, JRuby, Jython, and Kotlin. (JRuby and Jython are Java implementations of the Ruby and Python languages, respectively.) All of these languages can create and use Java objects. Hence, \pkg{jsr223} enables developers to easily embed other languages in R that can, in turn, use Java classes in natural syntax. As we will show, this approach is easier to use than \pkg{rJava}'s low-level JNI interface. In addition to multi-language support for the Java platform, \pkg{jsr223} provides an intuitive callback API to access R from any of the supported languages. Finally, the \pkg{jsr223} package simplifies and extends data exchange between R and Java. In all, \pkg{jsr223} lowers the barrier to the performance benefits and cross-platform capabilities of the Java platform.

There are three primary use cases for \pkg{jsr223}. Most importantly, any of the \pkg{jsr223} languages can be used to quickly embed Java solutions in R. Second, \pkg{jsr223} provides access to solutions developed in any of the \pkg{jsr223} languages. Lastly, the callback interface included with \pkg{jsr223} can be used to embed R within any of the \pkg{jsr223}-supported languages.

\subsection{Document organization}

We define some needed terminology, and then we put the \pkg{jsr223} project in context with a review of relevant software. Next, there are the necessary installation instructions and a quick start guide to demonstrate major package features. The most interesting sections follow with code samples that blend R with the \pkg{jsr223} languages. All code samples related to this document are available at \href{https://github.com/floidgilbert/jsr223}{GitHub}.

\section{Helpful terminology and concepts}

Java programs are compiled to Java bytecode that can be executed by an instance of a Java Virtual Machine (JVM). A JVM is an abstraction layer that provides a platform-independent execution environment for Java programs. A JVM interprets Java bytecode to machine code (i.e., processor-specific instructions). JVMs are available for a wide variety of hardware and software platforms. In principle, the same Java program will run on any platform that supports a JVM. The Java paradigm contrasts with traditional compiled languages, such as C, that are compiled directly to processor-dependent machine code, and therefore must be recompiled for every targeted architecture. Often, changes in the source code are also required to support different platforms.

Today, there are several programming languages that compile down to Java bytecode including all of the languages currently supported by \pkg{jsr223}. This may be surprising to some readers because languages like JavaScript are traditionally interpreted only, not compiled. In fact, the \pkg{jsr223} languages blur the line between scripting languages (those that are interpret-only) and traditional compiled languages. Nevertheless, we generally refer to the languages supported by \pkg{jsr223} as scripting languages in this document because, as far as the user is aware, source code is interpreted and executed (i.e., evaluated) in one step. Even so, this implementation benefits from the significant performance gains of compiled code.

A \dfn{scripting engine} (usually shortened to \dfn{script engine}) is software that enables a scripting language to be embedded in an application. Internally, a script engine uses an \dfn{interpreter} to parse and execute source code. The terms \dfn{script engine} and \dfn{interpreter} are often used interchangeably. In this document, \dfn{script engine} refers to the software component, not the interpreter. A \dfn{script engine instance} denotes an instantiated script engine. Finally, a \dfn{script engine environment} refers to the state (i.e., the variables and settings) of a given instance.

\dfn{Bindings} refers to the name/value pairs associated with variables in a given scope. Conceptually, a variable's name is bound to its value. The variable names and values in R's global environment are examples of bindings.

\hypertarget{softwarereview}{\section{Software review}}

There are many programming language integrations that combine the strengths of R with other development environments. These language integrations can generally be classified as either \dfn{R-major} or \dfn{R-minor}. R-major integrations use R as the primary environment to control some other embedded language environment. R-minor integrations are the inverse of R-major integrations. For example, \pkg{rJava} is an R-major integration that allows Java objects to be used within an R session. The Java/R Interface (\CRANpkg{JRI}), in contrast, is an R-minor integration that enables Java applications to embed R.

The \pkg{jsr223} package provides an R-major integration for the Java platform and the \pkg{jsr223} programming languages. In this software review, we provide context for the \pkg{jsr223} project through comparisons with other R-major integrations. Popular R-minor language integrations such as \CRANpkg{Rserve} \citep{rserve} and \CRANpkg{OpenCPU} \citep{opencpu} are not included in this discussion because their objectives and features do not necessarily align with those of \pkg{jsr223}.

Before we discuss individual packages, we point out one unique feature that contrasts \pkg{jsr223} with all other integrations in this discussion: \pkg{jsr223} is the only package that provides a standard interface to integrate R with multiple programming languages. This key feature enables developers to take advantage of solutions and features in several languages without the need to learn or install new packages.

\subsection{\pkg{rJava}}

As noted in the introduction, \pkg{rJava} is the preeminent Java integration for R. It provides a low-level interface to compiled Java classes via the JNI. The \pkg{jsr223} package uses \pkg{rJava} together with the Java Scripting API to create a simple, multi-language integration for R and the Java platform.

The following code example is taken from \pkg{rJava}'s web site \citep{rjavaweb}. It demonstrates the essential functions of the \textbf{rJava} API with code to create and display a window with a single button. The first two lines are required to initialize \pkg{rJava}. The next lines use the \code{.jnew()} function to create two Java objects: a GUI frame and a button. The associated class names are denoted in JNI syntax. Of particular note is the first invocation of \code{.jcall()}, the function used to call object methods. In this case, the \code{add()} method of the frame object is invoked. For \pkg{rJava} to identify the appropriate method, an explicit return type must be specified in JNI notation as the second parameter to \code{.jcall()} (unless the return value is \code{void}). The last parameter to \code{.jcall()} specifies the object to be added to the frame object. It must be explicitly cast to the correct interface for the call to be successful.

\begin{verbatim}
library("rJava")
.jinit()
f <- .jnew("java/awt/Frame", "Hello")
b <- .jnew("java/awt/Button", "OK")
.jcall(f, "Ljava/awt/Component;", "add", .jcast(b, "java/awt/Component"))
.jcall(f, , "pack")
# Show the window.
.jcall(f, , "setVisible", TRUE)
# Close the window.
.jcall(f, , "dispose")
\end{verbatim}

The snippet below reproduces the \pkg{rJava} example using JavaScript. In comparison, the JavaScript code is more natural for most programmers to write and maintain. The fine details of method lookups and invocation are handled automatically: no explicit class names or type casts are required. This same example can be reproduced in any of the five other \pkg{jsr223}-supported programming languages.

\begin{verbatim}
var f = new java.awt.Frame('Hello');
f.add(new java.awt.Button('OK'));
f.pack();
// Show the window.
f.setVisible(true);
// Close the window.
f.dispose();
\end{verbatim}

Using \pkg{jsr223}, the preceding code snippet can be embedded in an R script. The first step is to create an instance of a script engine. A JavaScript engine is created as follows:

\begin{verbatim}
library(jsr223)
engine <- ScriptEngine$new("JavaScript")
\end{verbatim}
This engine object is now ready to evaluate script on demand. Source code can be passed to the engine using character vectors or files. The sample below demonstrates embedding JavaScript code in-line with character vectors. This method is appropriate for small snippets of code. (Note: If you try this example the window may appear in the background. Also, the window must be closed using the last line of code. These are limitations of the code example, not \pkg{jsr223}.)

\begin{verbatim}
# Execute code inline to create and show the window.
engine %@% "
  var f = new java.awt.Frame('Hello');
  f.add(new java.awt.Button('OK'));
  f.pack();
  f.setVisible(true);
"

# Close the window
engine %@% "f.dispose();"
\end{verbatim}
To execute source code in a file, use the script engine object's \code{source} method: \code{engine\$source(file.name)}. The variable \code{file.name} may specify a local file path or a URL. % Whether evaluating small code snippets or sourcing script files, embedding source code using \pkg{jsr223} is straightforward. The ability to embed source code in R facilitates prototyping and promotes rapid application development.

In comparison to \pkg{rJava}'s low-level interface, \pkg{jsr223} allows developers to use Java objects without knowing the details of JNI and method lookups. However, it is important to note that \pkg{rJava} does include a high-level interface for invoking object methods. It uses the Java reflection API to automatically locate the correct method signature. This is an impressive feature, but according to the \pkg{rJava} web site, its high-level interface is experimental and does not work correctly for all scenarios. There are no such limitations for the \pkg{jsr223} scripting languages.

The \pkg{jsr223} languages also feature support for advanced object-oriented constructs. For example, classes can be extended and interfaces can be implemented using any language. These features allow developers to quickly implement sophisticated solutions in R without developing, compiling, and distributing custom Java classes.

The \pkg{rJava} package supports exchanging scalars, arrays, and matrices between R and Java. The following R code demonstrates converting an R matrix to a Java object, and vice versa, using \pkg{rJava}.

\begin{verbatim}
a <- matrix(rnorm(10), 5, 2)
# Copy matrix to a Java object with rJava
o <- .jarray(a, dispatch = TRUE)
# Convert it back to an R matrix.
b <- .jevalArray(o, simplify = TRUE)
\end{verbatim}

Again, the \pkg{jsr223} package builds on \pkg{rJava} functionality by extending data exchange. Our package converts R vectors, factors, n-dimensional arrays, data frames, lists, and environments to generic Java objects.\footnote{\pkg{rJava}'s interface can theoretically support n-dimensional arrays, but currently the feature does not produce correct results for n $>$ 2. See the related issue at the \pkg{rJava} Github repository: ``.jarray(..., dispatch=T) on multi-dimensional arrays creates Java objects with wrong content.''} In addition, \pkg{jsr223} can convert Java scalars, n-dimensional arrays, maps, and collections to base R objects. Several data exchange options are available, including row-major and column-major ordering schemes for data frames and n-dimensional arrays. %Many language integration packages employ JSON serialization to implement a similar feature set. Our package uses custom serialization routines to avoid the round-off error and overhead associated with JSON.
%///consider adding details for data exchange in the full article.

This code snippet demonstrates data exchange using \pkg{jsr223}. The variable \code{engine} is a \pkg{jsr223} ScriptEngine object. Similar to the preceding \pkg{rJava} example, this code copies a matrix to the Java environment and back again. The same syntax is used for all supported data types and structures.

\begin{verbatim}
a <- matrix(rnorm(10), 5, 2)
# Copy an R object to Java using jsr223.
engine$a <- a
# Retrieve the object.
engine$a
\end{verbatim}

%///not sure this is true about \pkg{JRI} when invoked from \pkg{rJava}. Find out. It could add or remove material here. Could consider showing a \pkg{jsr223} example.
The \pkg{rJava} package does not directly support callbacks into R. Instead, callbacks are implemented through \CRANpkg{JRI}: the Java/R Interface. To use \pkg{JRI}, R must be compiled with the shared library option \samp{-{}-enable-R-shlib}. \pkg{JRI} also requires several paths and environment variables to be set, so it includes a launch script to get started. The \pkg{JRI} interface is technical and extensive. In contrast, \pkg{jsr223} supports callbacks into R using a lightweight interface that provides just three methods to execute R code, set variable values, and retrieve variable values. The \pkg{jsr223} package does not use \pkg{JRI}, so there is no requirement for R to be compiled as a shared library.

In conclusion, \pkg{jsr223} provides an alternative integration for the Java platform that is easy to learn and use.

\subsection{JavaScript integrations}

\subsection{Groovy integrations}

\subsection{Python integrations}

\subsection{Other integrations}
Scala. Kotlin. Ruby. ///RScript?
Currently, \pkg{jsr223} is the only integration for the Ruby and Kotlin languages on CRAN.%///ok here? still true?

\section{Package installation}

The \pkg{jsr223} package requires Java 8 Standard Edition or above. The current version of the Java Runtime Environment (JRE) can be determined by executing \samp{java -version} from a system command prompt. See the example output below. Java 8 is denoted by version 1.8.x\_xx.

\begin{verbatim}
java version "1.8.0_91"
Java(TM) SE Runtime Environment (build 1.8.0_91-b15)
Java HotSpot(TM) 64-Bit Server VM (build 25.91-b15, mixed mode)
\end{verbatim}

The JRE can be obtained from  \href{http://www.oracle.com/technetwork/java/javase/downloads/jre8-downloads-2133155.html}{Oracle's web site}. Select the architecture (32 or 64 bit) that matches your R installation.

\pkg{jsr223} runs on a standard installation of R (e.g., the R build option \code{--enable-R-shlib} is not required). \pkg{jsr223} is available on CRAN and can be installed with the usual command:

\begin{verbatim}
install.packages("jsr223")
\end{verbatim}
This command will also download and install the \pkg{rJava} dependency necessary. However, the \pkg{rJava} installation will fail if R is not yet configured to use Java. To configure R for Java in Linux or OSX, execute \samp{sudo R CMD javareconf} in a terminal. For Windows, open a command prompt “As Administrator” and execute \samp{R CMD javareconf}. If the Java reconfiguration command generates errors, address the errors and execute the command again. One common error can be resolved by determining whether the GNU Compiler Collection (GCC) is accessible. To check for GCC, execute `gcc --help` from a terminal. This command will fail if GCC is not installed or if the license agreement has not been accepted.

\hypertarget{scriptengineinstallation}{\section{Script engine installation and instantiation}}

To create an instance of a language's script engine, \pkg{jsr223} requires access to the associated Java Archive (JAR) files. These instructions will help you obtain the required files and create a script engine instance.

\subsection{Groovy}

\href{http://groovy-lang.org}{Groovy} is a Java-like scripting language. Java code can often be executed by the Groovy engine with little modification. Hence, this Groovy integration essentially brings the Java language to R.

To obtain the standalone Groovy engine, go to \url{http://groovy-lang.org} and click the \samp{Download} link. Locate the current binary distribution. Download and extract the archive to a temporary folder. Locate the \samp{embeddable} subfolder. Copy the file named \samp{groovy-all-x.x.x.jar} to a convenient location and make note of the path. Specify this path in the \code{class.path} parameter of the \code{ScriptEngine\$new} constructor to create a Groovy script engine instance:

\begin{verbatim}
library("jsr223")
engine <- ScriptEngine$new("groovy", class.path = "~/your-path/groovy-all.jar")
\end{verbatim}

\subsection{JavaScript (Nashorn)}

\href{https://docs.oracle.com/javase/8/docs/technotes/guides/scripting/nashorn/}{Nashorn} is the JavaScript dialect included in Java 8. Nashorn implements ECMAScript 5.1. Because Nashorn is included in the JRE, no downloads are necessary to use JavaScript with \pkg{jsr223}. Technical documentation and examples for Nashorn are available at \href{https://docs.oracle.com/javase/8/docs/technotes/guides/scripting/nashorn/}{Oracle's web site}. Create a JavaScript instance using

\begin{verbatim}
library("jsr223")
engine <- ScriptEngine$new("javascript")
\end{verbatim}

\subsection{JRuby}

\href{http://jruby.org}{JRuby} is a Java-based implementation of the Ruby programming language. Obtain the standalone JRuby engine by clicking the \samp{Downloads} link at at \url{http://jruby.org}. Find \samp{JRuby x.x.x.x Complete.jar} and save it to a convenient location. Specify the path to the JAR file in the \code{class.path} parameter of the \code{ScriptEngine\$new} constructor to create a JRuby script engine instance.

\begin{verbatim}
library("jsr223")
engine <- ScriptEngine$new("ruby", class.path = "~/your-path/jruby-complete.jar")
\end{verbatim}

\subsection{Jython}

\href{http://www.jython.org}{Jython} is a Java-based implementation of the Python programming language. The standalone Jython engine is available at \url{http://www.jython.org}. Follow the \samp{Download} link. Click \samp{Download Jython x.x.x - Standalone Jar} to start the download. Save the JAR file to a convenient location and remember the path. This path will be used by \pkg{jsr223} to load the script engine as in the following code.

\begin{verbatim}
library("jsr223")
engine <- ScriptEngine$new("python", class.path = "~/your-path/jython-standalone.jar")
\end{verbatim}

\hypertarget{kotlinscriptengineinstallation}{\subsection{Kotlin}}

\href{https://kotlinlang.org/}{Kotlin} is a relatively new programming language that is interoperable with Java. As of this writing, a standalone JAR file is not available for the script engine. The most straight-forward way to obtain the files is to use selected files from the Community Edition of the \href{https://www.jetbrains.com/idea/}{JetBrains IntelliJ Idea} integrated development environment (IDE). The IDE doesn't need to be installed. Download the IDE's archive file (e.g., a zip file, not the executable installer package). Create an empty target folder on your system for the Kotlin files. Extract the \file{bin} and \file{plugins/Kotlin} folders to the target folder preserving the original folder structures. \strong{Note:} The \file{bin} folder isn't strictly required, but it will eliminate warnings on some systems. Make note of the fully-qualified path to the \file{plugins/Kotlin} folder; it will be used by \pkg{jsr223} to load the script engine.

If you are already using a current version of IntelliJ Idea, or if you decide to install the IDE, locate the path to the \file{plugins/Kotlin} subfolder of the IDE's installation path. This folder will be used to load the script engine.

Because Kotlin does not provide a standalone script engine JAR file, \pkg{jsr223} includes a convenience function \code{getKotlinScriptEngineJars()} to simplify adding JAR files to the class path. The following code demonstrates creating a Kotlin script engine instance using only the minimum required JAR files. The \code{kotlin.path} variable contains the path to the \file{plugins/Kotlin} folder on your system.

\begin{verbatim}
library("jsr223")
engine <- ScriptEngine$new(
  "kotlin",
  class.path = getKotlinScriptEngineJars(kotlin.path)
)
\end{verbatim}
To include all Kotlin system JAR files in the class path, use this example instead.

\begin{verbatim}
library("jsr223")
engine <- ScriptEngine$new(
  "kotlin",
  class.path = getKotlinScriptEngineJars(kotlin.path, minimum = FALSE)
)
\end{verbatim}

%Visit the Kotlin web site at \url{https://kotlinlang.org/} and follow the link to download the standalone compiler. Extract the compiler into a directory and make note of the path; it will be used to create a script engine instance. The compiler distribution does not include the required script engine file \file{kotlin-script-util-*.jar}. Perhaps the simplest way to obtain it is to use the Maven repository. Follow \href{https://mvnrepository.com/artifact/org.jetbrains.kotlin/kotlin-script-util}{this link}, or search for \samp{kotlin-script-util} at \url{https://mvnrepository.com}. Click the link corresponding your version of the Kotlin compiler. Click the link labeled \samp{jar} under the listed files. Save the JAR file to the \file{lib} subfolder of the Kotlin compiler installation directory.

\hypertarget{quickstartguide}{\section{Quick start guide}}

The primary features of \pkg{jsr223} are designed to be accessible to R programmers of all experience levels. This quick start guide illustrates these features with simple code examples. In general, the code samples work with all supported script engines with two exceptions.

\begin{enumerate}
\item Global variables in Ruby script must be prefixed with a dollar sign.
\item Kotlin script engine bindings are not created as global variables. See \hyperlink{kotlinlanguagedetails}{Kotlin language details}.
\end{enumerate}

\subsection{Hello world}

The R code snippet below demonstrates the basic elements required to embed a scripting language: start a script engine, optionally pass data to the script engine environment, execute a script, and terminate the script engine when it is no longer needed.

%///replace all verbatim with example or whatever it is that R Journal uses.
\begin{verbatim}
library("jsr223")
engine <- ScriptEngine$new("javascript")
engine$message <- "Hello world"
engine %~% "print(message);"

## Hello world

engine$terminate()
\end{verbatim}

The \code{ScriptEngine\$new} constructor method creates a script engine instance. In the preceding example, we assign the new script engine object to the variable \code{engine}. The first argument of \code{ScriptEngine\$new} specifies the type of script engine to create. In this case, we create a JavaScript engine. The third line assigns the value \code{"Hello world"} to a global variable named \code{message} in the script engine environment. The next line executes a JavaScript code snippet using the \code{\%$\sim$\%} operator. The snippet uses the JavaScript \code{print} method to write the message to the console. The last line in the example terminates the script engine and releases the associated resources.

To create a script engine other than JavaScript, specify a different script engine name and a character vector containing the required script engine JAR files. (See \hyperlink{scriptengineinstallation}{Script engine installation} for instructions to obtain script engines.) The supported script engine names are listed in Table \ref{tab:script-engine-type-names}. These names are defined by the script engine provider. \strong{Note:} Script engine names are case sensitive.

The next example reproduces the “Hello world” example in Ruby script.

\begin{verbatim}
library("jsr223")
engine <- ScriptEngine$new(
  engine.name = "ruby",
  class.path = "~/your-path/jruby-complete.jar"
)
engine$message <- "Hello world"
engine %~% "puts $message"

## Hello world

engine$terminate()
\end{verbatim}
In this case, two parameters are passed to the \code{ScriptEngine\$new} method: the script engine name \code{"ruby"}, and the path to the JRuby script engine JAR file. As before, we assign the value \code{"Hello world"} to a global variable named \code{message} and print it to the console. Notice that we prefix the global variable with a dollar sign: \code{\$message}. This syntax is peculiar to global variables in the Ruby language.

\begin{table}[h]
    \small
    \centering
    \begin{tabular}{l p{8cm}}
        \toprule
        \textbf{Language} & \textbf{Script engine names} \\
        \midrule
        \noalign{\vspace{1ex}}
        \href{http://groovy-lang.org}{Groovy} &  \code{groovy}, \code{Groovy}\\[.25cm]
        \href{https://docs.oracle.com/javase/8/docs/technotes/guides/scripting/nashorn/}{JavaScript (Nashorn)} & \code{js}, \code{JS}, \code{JavaScript}, \code{javascript}, {nashorn}, \code{Nashorn}, \code{ECMAScript}, \code{ecmascript}\\[.25cm]
        \href{http://jruby.org}{JRuby (Ruby)} & \code{jruby}, \code{ruby}\\[.25cm]
        \href{http://www.jython.org}{Jython (Python)} & \code{jython}, \code{python}\\[.25cm]
        \href{https://kotlinlang.org/}{Kotlin} & \code{kotlin}\\
        \noalign{\vspace{1ex}}
        \bottomrule
    \end{tabular}
    \caption{The \code{ScriptEngine\$new} constructor method creates a new script engine instance for a given language using the associated names in this table. Script engine names are case sensitive.}
    \label{tab:script-engine-type-names}
\end{table}

\subsection{Executing script}

\pkg{jsr223} provides several methods to execute script. The lines

\begin{verbatim}
return.value <- engine %~% script
return.value <- engine$eval(script)
\end{verbatim}
both evaluate the expression in the character vector \code{script}. The return value is the result of the last expression in the script, if any, or \code{NULL} otherwise. Text written to standard output by the script engine is printed to the R console. The following line executes JavaScript code and assigns the result to an R variable.

\begin{verbatim}
result <- engine %~% "isFinite(1);"
\end{verbatim}
The following lines also execute script, but there are no return values. This notation is convenient if the last expression in the snippet returns unneeded data or an unsupported type (like a function).

\begin{verbatim}
engine %@% script
engine$eval(script, discard.return.value = TRUE)
\end{verbatim}
To execute a script file, use either of the following lines where \code{file.name} is the path or URL to the script file.

\begin{verbatim}
engine$source(file.name)
engine$source(file.name, discard.return.value = TRUE)
\end{verbatim}

The methods \code{eval} and \code{source} take an argument named \code{bindings} that accepts an R named list. The name/value pairs in the list replace the script engine's global bindings during script execution. The following JavaScript example demonstrates this functionality. Notice that the result of \code{a + b} changes when bindings are specified.

\begin{verbatim}
engine$a <- 2
engine$b <- 3
engine$eval("a + b")

## 5

lst1 <- list(a = 6, b = 7)
engine$eval("a + b", bindings = lst1)

## 13
\end{verbatim}
This script would throw an error because 'b' is not defined in the list.
\begin{verbatim}
lst2 <- list(a = 6)
engine$eval("a + b", bindings = lst2)
\end{verbatim}
When the \code{bindings} parameter is not specified, the script engine reverts to the default global bindings.
\begin{verbatim}
engine$eval("a + b")

## 5
\end{verbatim}

\subsection{Sharing data between language environments}

The following two lines of R code are equivalent: they convert an R object to a Java object and assign the new object to a variable \code{myValue} in the script engine's environment. This syntax is the same for all supported R data structures.

\begin{verbatim}
engine$myValue <- iris
engine$set("myValue", iris)
\end{verbatim}

To retrieve \code{myValue} from the script engine (i.e., to convert a Java object to an R object), use either of the following lines.

\begin{verbatim}
engine$myValue
engine$get("myValue")
\end{verbatim}

Remove the \code{myValue} variable with \code{engine\$remove("myValue")}. List all bindings in the script engine's environment with \code{engine\$getBindings()}.

Bindings are synonymous with global variables in most script engine environments. For example, the following sample creates the binding \code{myValue} and retrieves it through JavaScript. This demonstrates that \code{myValue} is a global variable.

\begin{verbatim}
engine$myValue <- 5
engine %~% "myValue;"

## [1] 5
\end{verbatim}
The Kotlin language is an exception to this rule. It manages bindings through the global object \code{jsr223Bindings} as follows. See \hyperlink{kotlinlanguagedetails}{Kotlin language details} for more information.

\begin{verbatim}
engine$myValue <- 5
engine %~% 'jsr223Bindings["myValue"]'

## [1] 5
\end{verbatim}

All data structures in Java-based scripting languages are backed by Java objects. Discover the Java class for any global variable using  \code{engine\$getJavaClassName("identifier")} where \code{identifier} is the variable's name.

Behind the scenes, \pkg{jsr223}'s simplified data exchange is provided by \CRANpkg{jdx}: Java Data Exchange for R and \pkg{rJava}. The \pkg{jdx} package's functionality was originally part of \pkg{jsr223}, but it was broken out into a separate package to simplify maintenance and to make its features available to other developers.

The \pkg{jdx} package (and hence \pkg{jsr223}) supports converting R vectors, factors, n-dimensional arrays, data frames, named lists, unnamed lists, nested lists (i.e., lists containing lists), and environments to generic Java objects. Row-major and column-major ordering options are available for arrays and data frames. R data types numeric, integer, character, raw, and logical are supported. Complex types and date/time classes are not supported.

Java scalars, n-dimensional arrays, collections, and maps can be converted to standard objects in the R environment. These structures cover all of the primary data types in the supported scripting languages. Moreover, collections and maps are ubiquitous in Java APIs; providing support for these structures gives R developers easy access to a vast number of data structures available on the Java platform.

All \pkg{jdx} data conversion options are mirrored by settings in \pkg{jsr223}. The most pertinent details are discussed in the following sections. For a more thorough discussion, see the vignette included with the \pkg{jdx} package.

\subsection{Setting and getting script engine options}

The \pkg{jsr223} \code{ScriptEngine} class exposes several methods that control settings for a script engine instance. These methods are named using the Java getter/setter convention: methods that set values are prefixed with ``set'' and methods that retrieve values begin with ``get''. For example, if \code{engine} is a script engine object, \code{engine\$setArrayOrder('column-major')} will change the \textit{array order} setting. The code \code{engine\$getArrayOrder()} will retrieve the current \textit{array order} setting.

\hypertarget{handlingrvectors}{\subsection{Handling R vectors}}

By default, length-one R vectors are converted to Java scalars when passed to the script engine environment. If a Java length-one array is desired, wrap the value in the R ``as-is'' function (e.g., \code{I(myValue)}), or set the \textit{length one vector as array} setting to \code{TRUE} using the \code{setLengthOneVectorAsArray} method. By default, length-one vectors are converted to Java scalars as demonstrated here.

\begin{verbatim}
engine$setLengthOneVectorAsArray(FALSE)
engine$myScalar <- 1
engine$getJavaClassName("myScalar")

## [1] "java.lang.Double"
\end{verbatim}
Wrap a length-one vector with \code{I()} to indicate that an array should be created instead. In this case, the resulting Java class name is \code{"[D"} which denotes a primitive, double one-dimensional array.

To change the conversion behavior for all length-one vectors, set the \textit{length one vector as array} setting to \code{TRUE}.
\begin{verbatim}
engine$setLengthOneVectorAsArray(TRUE)
engine$myArray <- 1
engine$getJavaClassName("myArray")

## [1] "[D"
\end{verbatim}

Vectors of any length other than one are always converted to primitive Java arrays. The following code passes a vector of ten random normal deviates to the script engine environment. The first element of the resulting array is returned. \strong{Note:} Java arrays use zero-based indexes.

\begin{verbatim}
set.seed(10)
engine$norms <- rnorm(10)
engine %~% "norms[0]"

## [1] 0.01874617
\end{verbatim}

\subsection{Handling R matrices and other n-dimensional arrays}

By default, n-dimensional arrays are copied in row-major order. The following example demonstrates converting a simple 2 x 2 R matrix. Because the order is row-major, the last line of code returns the element in the first row, second column. Remember, Java arrays use zero-based indexes.

\begin{verbatim}
m <- matrix(1:4, 2, 2)
m

##      [,1] [,2]
## [1,]    1    3
## [2,]    2    4

engine$m <- m
engine %~% "m[0][1]"

## [1] 3
\end{verbatim}

%///update this based on what is changed in jdx.
%///furthermore, I might have the names all wrong. the data structures are correct, but the names might be wrong.
The \code{setArrayOrder} script engine method controls ordering for arrays converted from R to Java, and vice versa. Three array index ordering schemes are available: \code{'row-major'}, \code{'column-major'}, and \code{'column-minor'}. These settings control how the destination Java array is constructed.

Before describing the ordering schemes, it is helpful to think of n-dimensional arrays as collections of smaller structures. A one-dimensional array (a vector) is a collection of scalars. A two-dimensional array (a matrix) is a collection of one-dimensional arrays representing either rows or columns of the matrix. A three-dimensional array (a rectangular prism or cube) is a collection of matrices. A four-dimensional array is a collection of cubes, and so forth. 

Now we describe the each of the \textit{array order} settings.

\begin{itemize}
\item \code{'row-major'} -- The data of the resulting Java n-dimensional array are ordered \newline \code{[row][column][matrix]...[n]}. The \pkg{jsr223} package defaults to \code{'row-major'} because R syntax uses this indexing scheme (though R stores the array in memory using column-major order). This row-major scheme is not intuitive for Java programmers when n > 2 because Java n-dimensional arrays are constructed as high-order objects containing low-order objects.

\item \code{'column-major'} -- The data of the resulting Java n-dimensional array are ordered \newline \code{[n]...[matrix][column][row]}. This ordering scheme is natural for Java programmers: the data contained in the one-dimensional arrays represent columns of the parent matrix.

\item \code{'column-minor'} -- The data of the resulting Java n-dimensional array are ordered \newline \code{[n]...[matrix][row][column]}. This provides Java programmers with a natural ordering scheme where the arrays at the one-dimensional level represent rows of the parent matrix. For matrices, \code{'column-minor'} and \code{'row-major'} are equivalent.

\end{itemize}

\strong{Note:} If an R array is converted to Java using a particular array order, use the same array order when converting it back from Java to R. Otherwise, the data will be in the wrong order.

In the following JavaScript example, a three-dimensional array is copied to the script engine using each of the three indexing options. We use the Java static method \code{deepToString} to create a string representation of the array that shows the resulting order of the data in the script engine.

\begin{verbatim}
a <- array(1:8, c(2, 2, 2))
a

## , , 1
##
##      [,1] [,2]
## [1,]    1    3
## [2,]    2    4
##
## , , 2
##
##      [,1] [,2]
## [1,]    5    7
## [2,]    6    8

engine$setArrayOrder("row-major")
engine$a <- a
engine %~% "java.util.Arrays.deepToString(a);"

## [1] "[[[1, 5], [3, 7]], [[2, 6], [4, 8]]]"

engine$setArrayOrder("column-major")
engine$a <- a
engine %~% "java.util.Arrays.deepToString(a);"

## [1] "[[[1, 2], [3, 4]], [[5, 6], [7, 8]]]"

engine$setArrayOrder("column-minor")
engine$a <- a
engine %~% "java.util.Arrays.deepToString(a);"

## [1] "[[[1, 3], [2, 4]], [[5, 7], [6, 8]]]"
\end{verbatim}

\hypertarget{handlingrdataframes}{\subsection{Handling R data frames}}

R data frames can be converted to the script engine using either row-major or column-major order. Row-major order (the default) creates a list of records. This representation is perhaps the most common in programming for tabular data. Column-major order, on the other hand, creates a list of columns. Column-major structures are faster to create and are generally preferred for aggregate column calculations. Change the \textit{data frame order} setting with the \code{setDataFrameRowMajor} method.

When the row-major setting is selected (i.e., \code{engine\$setDataFrameRowMajor(TRUE)}), an R data frame is converted to a \href{https://docs.oracle.com/javase/8/docs/api/java/util/ArrayList.html}{\code{java.util.ArrayList}} object. The list contains  \href{https://docs.oracle.com/javase/8/docs/api/java/util/LinkedHashMap.html}{\code{java.util.LinkedHashMap}} objects that represent the rows of the data frame. Each member of the hash map is a name/value pair of a single field in the data frame. The name of the field is the corresponding column's name. The following example uses R's built-in \code{iris} data set to illustrate using row-major data frames in the script environment.

\begin{verbatim}
engine$setDataFrameRowMajor(TRUE)
engine$iris <- iris

# Return the number of rows.
engine %~% "iris.size()"

## [1] 150

# Retrieve the sepal length in the first row.
engine %~% "iris[0].get('Sepal.Length')"

## [1] 5.1

# Retrieve the second row as a list.
engine %~% "iris[1]"

## $`Sepal.Length`
## [1] 4.9
##
## $Sepal.Width
## [1] 3
##
## $Petal.Length
## [1] 1.4
##
## $Petal.Width
## [1] 0.2
##
## $Species
## [1] "setosa"
\end{verbatim}

When the column-major setting is selected (i.e., \code{engine\$setDataFrameRowMajor(FALSE)}), an R data frame is converted to a \href{https://docs.oracle.com/javase/8/docs/api/java/util/LinkedHashMap.html}{\code{java.util.LinkedHashMap}} object. The map members are arrays representing the columns in the data frame.

Row names for data frames are not preserved during conversion. To include row names in the conversion, simply add them as a column in your data frame. We do not automatically include row names in conversion because it would require us to create an additional element in the Java map with a reserved key value such as \code{\_row}. Instead, we leave the decision of how to handle row names to the developer.

The following commented example uses R's built-in \code{mtcars} data set to illustrate basic functionality.

\begin{verbatim}
engine$setDataFrameRowMajor(FALSE)

# 'mtcars' is an R data frame containing information for 32 cars. 'mtcars'
# stores vehicle names as row names. Row names are not preserved during
# conversion. This line creates a new R data frame with the vehicle names as
# a new column 'name'.
df <- data.frame(name = row.names(mtcars), mtcars)

# This line converts the new data frame to a Java map named 'mtcars'.
engine$mtcars <- df

# Return the number of columns in the map.
engine %~% "mtcars.size()"

## [1] 12

# Access each column using the map's 'get' method and the column's name. This
# line returns the first element of the column 'name'.
engine %~% "mtcars.get('name')[0]"

## [1] "Mazda RX4"

# Add a new column named 'cylsize' representing the size of a single cylinder.
engine$cylsize <- mtcars[, "disp"] / mtcars[, "cyl"]
engine %@% "mtcars.put('cylsize', cylsize)"

# Remove the columns 'name' and 'cylsize'.
engine %@% "mtcars.remove('name')"
engine %@% "mtcars.remove('cylsize')"

# Compare the contents of the map to the original data frame in R.
all.equal(mtcars, engine$mtcars, check.attributes = FALSE)

## [1] TRUE
\end{verbatim}

Groovy and JavaScript support an additional syntax that allows map elements to be accessed like object properties instead of using the \code{get} and \code{put} methods.

\begin{verbatim}
# The following two lines are equivalent in Groovy and JavaScript.
engine %~% "mtcars.cyl[0];"
engine %~% "mtcars.get('cyl')[0];"
\end{verbatim}

\subsection{Handling R factors}

R factors are comprised of a character vector of levels and an integer vector of indexes that reference the levels. For example, if the integer vector \code{5:7} is converted to a factor, the levels will be \code{c("5", "6", "7")} and the indexes will be \code{c(1L, 2L, 3L)}. The script engine \textit{coerce factors} setting determines how the factor levels are handled when converting the factor to a Java array. When this setting is enabled (e.g., \code{engine\$setCoerceFactors(TRUE)}), an attempt is made to coerce the factor levels to integer, numeric, or logical values. If coercion fails, the character levels are used. When \textit{coerce factors} is disabled, the factor is always converted to a string array. The \textit{coerce factors} setting applies to standalone factors as well as factors in data frames.

After \pkg{jsr223} converts an R factor to a Java array, there is no consistent way to determine whether the array was originally created from an R factor. Therefore, if an R factor is copied to the script engine, and then the resulting array is returned to R, it will be converted to an R vector, not a factor.

When creating a data frame in R, character vectors are converted to factors by default. The \pkg{jsr223} package follows this standard when a qualifying Java object is converted to an R data frame. The \code{setStringsAsFactors} method modifies this behavior. The method takes one of three values: \code{NULL}, \code{TRUE}, and \code{FALSE}. If \code{NULL} is specified (the default), the R system setting is used (see \newline\code{getOption("stringsAsFactors")}). A value of \code{TRUE} ensures that character vectors are always converted to factors for new data frames. Finally, a setting of \code{FALSE} disables conversion to factors.

\subsection{Handling R lists and environments}
The \pkg{jsr223} package converts R lists and environments to Java objects. List elements may be any R data structure supported by \pkg{jsr223}, including other lists (i.e., nested lists). There is no limitation to the levels of nesting.

R named lists and environments are converted to Java \href{https://docs.oracle.com/javase/8/docs/api/java/util/HashMap.html}{\code{java.util.HashMap}} objects. See \hyperlink{handlingrdataframes}{Handling R data frames} for map code examples. The only difference is that a data frame's contents are always converted to a map of arrays. For lists, the map elements may be any data structure.

R unnamed lists are converted to Java objects implementing the \href{https://docs.oracle.com/javase/8/docs/api/java/util/ArrayList.html}{\code{java.util.ArrayList}} interface. The following code demonstrates basic \code{java.util.ArrayList} functionality.

\begin{verbatim}
# Create an unnamed list with three elements.
engine$list <- list(c("a", "b", "c"), TRUE, pi)

# Members in the list are accessed by index. This line returns the first element.
engine %~% "list[0]"

## [1] "a" "b" "c"

# Replace an element in the list.
engine %@% "list[0] = 'replaced'"

# Add a new element to the end of the list.
engine %@% "list.add('last item')"

# Insert a new item before the first item.
engine %@% "list.add(0, 'first item')"

# Remove the last item.
engine %@% "list.remove(list.size() - 1)"

# Return the number of elements
engine %~% "list.size()"

## [1] 4
\end{verbatim}

\subsection{Data exchange details}

So far, we have discussed all of the basic functionality and settings related to data exchange. This section includes a few additional notes for data exchange. A comprehensive guide, including details for unexpected conversion behaviors, is included in the \pkg{jdx} package vignette.

R reserves special \code{NA} values to indicate missing types. Table \ref{tab:r-na-behavior} outlines how \code{NA} values are handled for different R data types. Table \ref{tab:java-null-behavior}, in turn, describes how Java null values are interpreted when converting Java objects to R.

\begin{table}[b]
\centering
\begin{tabular}{@{}ll@{}}
\toprule
R Structure      & NA Behavior                                         \\ \midrule
\code{numeric}   & \code{NA\_real\_} maps to a reserved value.           \\[.25cm]
\code{integer}   & \code{NA\_integer\_ maps} to a reserved value.        \\[.25cm]
\code{character} & \code{NA\_character\_} maps to Java \code{null}.      \\[.25cm]
\code{logical}   & \code{NA} maps to Java \code{false} with a warning. \\ \bottomrule
\end{tabular}
\caption{R reserves special \code{NA} values to indicate missing types. This table outlines how \code{NA} values are converted to Java values.}
\label{tab:r-na-behavior}
\end{table}

\begin{table}[t]
\centering
\begin{tabular}{@{}ll@{}}
\toprule
Java Structure                             & Java null Conversion   \\ \midrule
\code{Boolean[]..[]} & \code{null} maps to \code{FALSE} with a warning.  \\[.25cm]
\code{Byte[]..[]}       & \code{null} maps to \code{raw} \code{0x00} with a warning. \\[.25cm]
\code{Character[]..[]}  & \code{null} maps to \code{NA\_character\_}.   \\[.25cm]
\code{Double[]..[]}   & \code{null} maps to \code{NA\_real\_}.          \\[.25cm]
\code{Float[]..[]}     & \code{null} maps to \code{NA\_real\_}.        \\[.25cm]
\code{Integer[]..[]}     & \code{null} maps to \code{NA\_integer\_}.                  \\[.25cm]
\code{java.math.BigDecimal[]..[]}          & \code{null} maps to \code{NA\_real\_}.                     \\[.25cm]
\code{java.math.BigInteger[]..[]}          & \code{null} maps to \code{NA\_real\_}.                     \\[.25cm]
\code{Long[]..[]}       & \code{null} maps to \code{NA\_real\_}.                     \\[.25cm]
\code{Object[]..[]}                        & \code{null} maps to \code{NULL}.                           \\[.25cm]
\code{Short[]..[]}     & \code{null} maps to \code{NA\_integer\_}.                  \\[.25cm]
\code{java.lang.String[]..[]}              & \code{null} maps to \code{NA\_character\_}.                \\ \bottomrule
\end{tabular}
\caption{Java \code{null} indicates missing or uninitialized values. This table outlines how \code{null} is interpreted when converting Java objects to R. The syntax \code{[]..[]} is used to indicate an array of one or more dimensions.}
\label{tab:java-null-behavior}
\end{table}

Because \pkg{jsr223} converts data to generic Java data structures, R attributes such as names cannot always be included in conversion. For example, R vectors are converted to native Java arrays, therefore names associated with vector elements must be discarded. Likewise, dimension names are not preserved for n-dimensional structures. Column names for data frames are preserved, but row names are not. To preserve data frame row names, simply copy the names to a new column before converting the data frame.

The \pkg{jsr223} package always converts R vectors and arrays to Java arrays. Java arrays are intuitive to use in all of the supported scripting environments. However, the supported scripting languages can also create array structures that are not native Java arrays. \pkg{jsr223} also supports converting these language-specific array and collection structures to R vectors and arrays.

Java n-dimensional arrays whose subarrays of a given dimension are not the same dimension are known as \dfn{ragged arrays}. Ragged arrays cannot be converted to R arrays. Instead, \pkg{jsr223} translates ragged arrays to lists of the appropriate object. For example, a matrix containing subarrays of different lengths will be converted to an R list of vectors. Likewise, a three-dimensional array containing two matrices of different dimensions will be converted to an R list of matrices.

As described earlier, R unnamed lists are converted to \href{https://docs.oracle.com/javase/8/docs/api/java/util/ArrayList.html}{\code{java.util.ArrayList}} objects. The \code{ArrayList} class implements the \href{https://docs.oracle.com/javase/8/docs/api/java/util/Collection.html}{\code{java.util.Collection}} interface. This is one of the most basic interfaces in Java and it is common to a large number of structures. \pkg{jsr223} converts Java objects implementing the \code{java.util.Collection} interface to vectors, n-dimensional arrays, data frames, and unnamed lists, depending on the structure's content. In some cases an R list converted to a Java object, and then converted back to an R object, may not produce an R list. See the sections ``Java Collections'' and ``Conversion Issues'' in the \pkg{jdx} package vignette for conversion rules and in-depth explanations.

The jdx package converts R raw values to Java byte values and vice versa. R raw values and Java byte values are both 8 bits, but they are interpreted differently. R raw values range from 0 to 255 (i.e., unsigned bytes). Java byte values range from -128 to 127 (i.e., signed bytes). The 8-bit value 0xff represents 255 in R, but is -1 in Java. Usually this discrepancy is not an issue because raw and byte values are used to store and transfer binary data such as images. If human-readable values are important, use integer vectors instead.

\subsection{Calling script functions and methods}

Functions and methods defined in script can be called directly from R via the \code{invokeFunction} and \code{invokeMethod} script engine methods. Any number of supported R structures can be passed as parameter values.

\strong{Note:} The Groovy, Python, and Kotlin engines can use \code{invokeMethod} to call methods of Java objects. The JavaScript and Ruby engines only support calling methods of native scripting objects.

As described in \hyperlink{handlingrvectors}{Handling R vectors}, length-one vectors are converted to Java scalars by default. One way to ensure that a vector is always converted to a Java array is by wrapping it in the ``as-is'' function \code{I()}. This feature is particularly useful when passing multiple parameters to a script function. In the same function, some parameters may require scalars while others require arrays. Simply use \code{I()} to indicate which vectors should be converted to arrays.

The following example demonstrates calling a simple JavaScript function, \code{sumThis}, that sums the elements of an array. If the first parameter is not an array, the function throws an error.

\begin{verbatim}
# Define a simple global function 'sumThis'.
engine %@% "
function sumThis(a) {
  if (!a.getClass().isArray())
    throw 'Not an array.';
  sum = 0;
  for (i = 0; i < a.length; i++) {
      sum += a[i];
  }
  return sum;
}
"

# Set the default length-one vectors setting so the example works as intended.
engine$setLengthOneVectorAsArray(FALSE)

# Call the function with a vector with length > 1.
vector <- c(1, 2, 3)
engine$invokeFunction("sumThis", vector)

## [1] 6

# If the vector is length-one, an error is thrown because an array parameter
# is expected.
vector <- 1
engine$invokeFunction("sumThis", vector)

## javax.script.ScriptException: Not an array. in <eval> at line number 4 at
## column number 4

# Try again, this time marking the vector as-is, meaning that it should
# always be converted to an array.
vector <- 1
engine$invokeFunction("sumThis", I(vector))

## [1] 1
\end{verbatim}

The next example demonstrates using \code{invokeMethod}. It is essentially the same as \code{invokeFunction} except that the first two parameters require the object's name and method, respectively.

\begin{verbatim}
# Invoke the 'abs' (absolute value) method of the JavaScript 'Math' object.
engine$invokeMethod("Math", "abs", -3)

## [1] 3
\end{verbatim}

\subsection{String interpolation}

\pkg{jsr223} features string interpolation before code evaluation. R code placed between \code{@\{} and \code{\}} in a code snippet is evaluated and replaced by the a string representation of the return value before the snippet is executed by the script engine. A script may contain multiple \code{@\{...\}} expressions. String interpolation is enabled by default. It can be disabled using \code{engine\$setInterpolation(FALSE)}.

\strong{Note:} Interpolated decimal values may lose precision when coerced to a string.

This example simply sums two numbers. The section \hyperlink{callbacks}{Callbacks} includes a more interesting interpolation example involving recursion.

\begin{verbatim}
a <- 1; b <- 2
engine %~% "@{a} + @{b}"

## 3
\end{verbatim}

Interpolation expressions are evaluated in the current scope. The following example shows that interpolation locates the value defined in the function's scope before the global variable of the same name.

\begin{verbatim}
a <- 1

constantFunction <- function() {
  a <- 3
  engine %~% "@{a}"
}

constantFunction()

## [1] 3
\end{verbatim}

\hypertarget{callbacks}{\subsection{Callbacks}}

Embedded scripts can access the R environment using the \pkg{jsr223} callback interface. When a script engine is started, \pkg{jsr223} creates a global object named \code{R} in the script engine's environment. This object is used to execute R code and set/get variables in the R session's global environment.

This code example demonstrates setting and getting a variable in the R environment. For Ruby, remember to prefix the global variable \code{R} with a dollar sign.

\begin{verbatim}
engine %@% "R.set('a', [1, 2, 3])"
engine %~% "R.get('a')"

## [1] 1 2 3
\end{verbatim}

\strong{Note:} Changing any of the data exchange settings will affect the behavior of the callback interface. For example, using \code{engine\$setLengthOneVectorAsArray(TRUE)} will cause \code{R.get("pi")} to return an array with a single element instead of a scalar value.

Execute R script with \code{R.eval(script)} where \code{script} is a string containing R code. This example returns a single random normal draw from R.

\begin{verbatim}
set.seed(10)
engine %~% "R.eval('rnorm(1)')"

## [1] 0.01874617
\end{verbatim}

Infinite recursive calls between R and the script engine are supported. The only limitation is available stack space. The following code demonstrates recursive calls and string interpolation with a countdown.

\begin{verbatim}
recursiveCountdown <- function(start.value) {
  cat("T minus ", start.value, "\n", sep = "")
  if (start.value > 0)
    engine %~% "R.eval('recursiveCountdown(@{start.value - 1})');"
}

engine %~% "R.eval('recursiveCountdown(3)')"

## T minus 3
## T minus 2
## T minus 1
## T minus 0
\end{verbatim}

\subsection{Embedding R in another scripting language}

It is often desirable to use R as an embedded language. The \pkg{jsr223} interface does not provide a standalone interface to call into R. However, the same functionality can be achieved with the \code{RScript} command line executable, a simple launch script, and the \pkg{jsr223} callback interface. The following R script is an example of a launch script for Groovy. It executes any Groovy script file provided as a command line parameter.

\begin{verbatim}
library("jsr223")
engine <- ScriptEngine$new("groovy", "~/my-path/groovy-all.jar")
tryCatch (
  engine$source(commandArgs(TRUE)[1], discard.return.value = TRUE),
  error = function(e) { cat(e$message, "\n", sep = "") },
  finally = { engine$terminate() }
)
\end{verbatim}

The following command line uses the launch script to execute a Groovy script. The launch script is named \samp{groovy-launcher.R} and \samp{source.groovy} is an arbitrary Groovy source file.

\begin{verbatim}
RScript groovy-launcher.R source.groovy
\end{verbatim}

With this setup, a developer can author a Groovy script in a dedicated script editor. The Groovy script can embed R using the \pkg{jsr223} callback interface as if it were a standalone interface. The command line above can be provided to a code editor to execute the Groovy script on demand. The Groovy code below is an example of embedding R.

\begin{verbatim}
// Set a variable named 'probabilities' in the R global environment.
R.set('probabilities', [0.25, 0.5, 0.20, 0.05]);

// Take a random draw of size two using the given probabilities.
draws = R.eval('sample(4, 2, prob = probabilities)');
\end{verbatim}

\subsection{Compiling script}

///mention that functions/methods are already compiled to bytecode.

The Java Scripting API supports compiling script to Java bytecode before evaluation. If unstructured code (i.e., code not encapsulated in methods or functions) is to be executed repeatedly, compiling it will improve performance. This feature does not apply to methods and functions as they are compiled on demand.

The following two lines show how to compile code snippets and source files, respectively. For the latter, local disk files or URLs can be specified. In both cases, a compiled script object is returned.

\begin{verbatim}
cs <- engine$compile(script)
cs <- engine$compileSource(file.name)
\end{verbatim}

The compiled script object has a single method, \code{eval}, that is used to execute the compiled code. It can be argued that the method should be called \code{exec} in this case, but our interface follows the Java Scripting API naming scheme. The following trivial example demonstrates the compiled script interface.

\begin{verbatim}
# Compile a code snippet.
cs <- engine$compile("c + d")

# This line would throw an error because 'c' and 'd' have not yet been declared.
## cs$eval()

engine$c <- 2
engine$d <- 3
cs$eval()

## 5
\end{verbatim}

The \code{eval} method takes an argument named \code{bindings} that accepts an R named list. The name/value pairs in the list replace the script engine's global bindings during script execution as shown in this code sample.
\begin{verbatim}
lst <- list(c = 6, d = 7)
cs$eval(bindings = lst)

## 13

# When 'bindings' is not specified, the script engine reverts to the original
# environment.
cs$eval()

## 5
\end{verbatim}

The \code{discard.return.value} argument of the \code{eval} method determines whether the return value of a script is discarded. The default is \code{FALSE}. The following line executes code but does not return a value.
\begin{verbatim}
cs$eval(discard.return.value = TRUE)
\end{verbatim}

\subsection{Handling console output}

When script is evaluated, any text printed to standard output appears in the R console by default. Console output can be disabled entirely with \code{engine\$setStandardOutputMode('quiet')}. To resume printing output to the console, use \code{engine\$setStandardOutputMode('console')}.

Text printed to the console by a script engine cannot be captured using R's \code{sink} or \code{capture.output} methods. To capture output, set the \textit{standard output mode} setting to \code{'buffer'}. In this JavaScript example, the \code{print} method output will not appear in the R console; it will be stored in an internal buffer. The contents of the buffer can be retrieved and cleared using the \code{getStandardOutput} method.

\begin{verbatim}
engine$setStandardOutputMode("buffer")
engine %@% ("print('abc');")
engine$getStandardOutput()

## [1] "abc\n"
\end{verbatim}
Alternatively, the buffer can be discarded using the \code{clearStandardOutput} method.
\begin{verbatim}
engine %@% ("print('abc');")
engine$clearStandardOutput()
\end{verbatim}

\subsection{Console mode: a simple REPL}

\pkg{jsr223} provides a simple read-evaluate-print-loop (REPL) for interactive code execution. This feature is inspired by Jeroen Ooms's \pkg{V8} package. The REPL is useful for quickly setting and inspecting variables in the script engine. Returned values are printed to the console using \code{base::dput}. The \code{base::cat} function is not used because it does not handle complex data structures.

Use \code{engine\$console()} to enter the REPL. Enter \samp{exit} to return to the R prompt. The REPL supports only single line entry: no line continuations or carriage returns are allowed. This limitation arises from the fact that the Java Scripting API does not support code validation.

The following output was produced by a Python REPL session. The code creates a Python dictionary object and accesses the elements. The tilde character (\samp{$\sim$}) indicates a prompt.

\begin{verbatim}
python console. Press ESC, CTRL + C, or enter 'exit' to exit the console.
~ dict = {"first": 1, "second": 2}

~ dict["first"]
1
~ dict["second"]
2
~ exit
Exiting console.
\end{verbatim}

Most developers are familiar with the command history in the R REPL.  Unfortunately, command history for the \pkg{jsr223} REPL is unreliable or non-existent because there is no functional standard for saving and restoring commands in R consoles.

\section{R with Groovy}

\href{http://www.groovy-lang.org/}{Groovy} is a dynamically typed programming language that closely follows Java syntax. Hence, the \pkg{jsr223} integration for Groovy enables developers to essentially embed Java language solutions in R. There are some minor language differences; they are described in the online guide  \href{http://groovy-lang.org/differences.html}{Differences with Java}. %///cite?.

\subsection{Groovy language details}

One syntactical detail relevant to \pkg{jsr223} users applies to global (top-level) variables. Top-level variables created in script will not persist in the script engine environment unless they are explicitly declared as global variables. Global variables are defined by omitting the type definition and Groovy's \code{def} keyword. For example \code{myValue = 42} will create a global variable. The \code{@myValue} notation cannot be used. To specify a data type for a global variable, use a constructor (\code{myVar = new Integer(42)}) or a type suffix (\code{myVar = 42L}).

\subsection{Groovy and Java Classes}

If you already know Java, using Java classes in Groovy will be very familiar. Java package members are imported (i.e., made accessible to the script) using the \code{import} statement. Groovy automatically imports many common Java packages by default such as \code{java.io.*}, \code{java.lang.*}, \code{java.net.*}, and \code{java.util.*}. If the package is not part of the JRE, add the package's JAR file to the \code{class.path} parameter of the \code{ScriptEngine\$new} constructor. \strong{Tip:} Supply class paths as separate elements of a vector instead of concatenating the paths with the usual path delimiters (“;” for Windows, and “:” for all others). This will make your code platform-independent and easier to read.

This example uses the \href{http://commons.apache.org/proper/commons-math/}{Apache Commons Mathematics Library} ///cite? to sample from a bivariate normal distribution.

\begin{verbatim}
library("jsr223")

# Include both the Groovy script engine and the Apache Commons Mathematics
# libraries in the class path. Specify the paths seperately in a character
# vector.
engine <- ScriptEngine$new(
  engine.name = "groovy",
  class.path = c("~/my-path/groovy-all.jar", "../commons-math3-3.6.1.jar")
)

# The getClassPath method displays the current class path.
engine$getClassPath()

# Define the means and covariance matrix that will be used to create the
# bivariate normal distribution.
engine$means <- c(0, 2)
engine$covariances <- diag(1, nrow = 2)

# Import the package member and instantiate a new class. For Groovy, excluding
# the type and 'def' keyword will make 'mvn' a global variable.
engine %@% "
  import org.apache.commons.math3.distribution.MultivariateNormalDistribution;
  mvn = new MultivariateNormalDistribution(means, covariances);
"

# Take a sample.
engine$invokeMethod("mvn", "sample")

## [1] 0.3279374 0.8652296

# Take three samples.
replicate(3, engine$invokeMethod("mvn", "sample"))

##           [,1]      [,2]      [,3]
## [1,] 0.9924368 -1.295875 0.2025815
## [2,] 2.5145855  2.128243 1.1666272

engine$terminate()
\end{verbatim}

\section{R with JavaScript}

The popularity of JavaScript has overflowed the arena of web development into standalone solutions involving databases, charting, machine learning, and network-enabled utilities, to name just a few. Many of these solutions can be harnessed by R with the help of \pkg{jsr223}. Even browser-based scripts that require a document object model (DOM) can be executed using Java's \code{WebView} browser. ///(See ///“A Groovy web UI for editing R data frames” for \code{WebView} details.) Popular JavaScript solutions can be found at \href{https://www.javascripting.com/}{JavaScripting}, an online database of JavaScript solutions. \href{https://github.com/trending/javascript?since=monthly}{Github} also lists trending solutions for JavaScript, as well as other languages.

\href{https://docs.oracle.com/javase/8/docs/technotes/guides/scripting/nashorn/}{Nashorn} is the JavaScript dialect included in Java 8. Nashorn implements ECMAScript 5.1. No download is required to use JavaScript with \pkg{jsr223}. Technical documentation and examples are available at \href{https://docs.oracle.com/javase/8/docs/technotes/guides/scripting/nashorn/}{Oracle's Nashorn web site} ///cite?.

\subsection{JavaScript and Java classes}

JavaScript Nashorn provides wide support for Java classes, including the ability to extend classes and implement interfaces. Nashorn provides several methods to reference JavaScript classes. The two most common methods are demonstrated below. This code uses a static class method to sort a vector of integers.

\begin{verbatim}
# The method recommended by Nashorn: create a reference to a Java class using
# the built-in Java.type() method. Conceptually, this is similar to importing
# the class.

engine %~% "
var Arrays = Java.type('java.util.Arrays');
var random = R.eval('sample(5)');
Arrays.sort(random);
random;
"

## [1] 1 2 3 4 5

# An alternative method uses the a fully qualified class name, but it
# requires more overhead per call.

engine %~% "
var random = R.eval('sample(5)');
java.util.Arrays.sort(random);
random;
"

## [1] 1 2 3 4 5
\end{verbatim}

The \code{Java.type()} method is required to create Java primitives as demonstrated in this example:

\begin{verbatim}
# Create a Java integer array with five elements.
var IntegerArrayType = Java.type('int[]');
var myArray = new IntegerArrayType(5);
\end{verbatim}

We reproduce the bivariate normal example in JavaScript to demonstrate importing an external library and to point out an important limitation in Nashorn regarding \code{invokeMethod}.

\begin{verbatim}
library("jsr223")

# Include the Apache Commons Mathematics library in class.path.
engine <- ScriptEngine$new(
  engine.name = "js",
  class.path = "../commons-math3-3.6.1.jar"
)

# Define the means and covariance matrix that will be used to create the
# bivariate normal distribution.
engine$means <- c(0, 2)
engine$covariances <- diag(1, nrow = 2)

# Import the package member and instantiate a new class.
engine %@% "
var MultivariateNormalDistributionClass = Java.type(
  'org.apache.commons.math3.distribution.MultivariateNormalDistribution'
);
var mvn = new MultivariateNormalDistributionClass(means, covariances);
"

# This line would throw an error. Nashorn JavaScript supports 'invokeMethod' for
# native JavaScript objects, but not for Java objects.
#
## engine$invokeMethod("mvn", "sample")

# Instead, use script...
engine %~% "mvn.sample();"

## [1] 0.3279374 0.8652296

# ...or wrap the method in a JavaScript function.
engine %@% "function sample() {return mvn.sample();}"
engine$invokeFunction("sample")

## [1] 0.2527757 1.1942332

# Take three samples.
replicate(3, engine$invokeFunction("sample"))

##           [,1]      [,2]      [,3]
## [1,] 0.9924368 -1.295875 0.2025815
## [2,] 2.5145855  2.128243 1.1666272

engine$terminate()
\end{verbatim}

\subsection{Using JavaScript solutions - Voca}

\href{https://vocajs.com/}{Voca} ///cite? is a popular library that simplifies many difficult string manipulation tasks such as word wrapping and diacritic detection (e.g., the “\'{e}” caf\'{e}). Using Voca with \pkg{jsr223} is simply a matter of sourcing a single script file. This sample script loads Voca and demonstrates its functionality.

\begin{verbatim}
# Source the Voca library. This creates a utility object named 'v'.
engine$source("./voca.min.js", discard.return.value = TRUE)

# 'prune' truncates string, without break words, ensuring the given length, including
# a trailing "..."
engine %~% "v.prune('A long string to prune.', 12);"

## [1] "A long..."

# Provide a different suffix to 'prune'.
engine %~% "v.prune('A long string to prune.', 16, ' (more)');"

## [1] "A long (more)"

# Voca supports method chaining.
engine %~% "
v('Voca chaining example')
  .lowerCase()
  .words()
"

## [1] "voca"     "chaining" "example"

# Split graphemes.
engine %~% "v.graphemes('cafe\u0301');"

## [1] "c" "a" "f" "é"

# Word wrapping.
engine %~% "v.wordWrap('A long string to wrap', {width: 10});"

## [1] "A long\nstring to\nwrap"

# Word wrapping with custom delimiters.
engine %~% "
v.wordWrap(
  'A long string to wrap',
  {
    width: 10,
    newLine: '<br/>',
    indent: '__'
  }
);
"

## [1] "__A long<br/>__string to<br/>__wrap"
\end{verbatim}

\section{R with Python}

Like R, the \href{https://www.python.org/}{Python} programming language is used widely in science and analytics. Python has a remarkably broad set of features, yet it is also known for being concise and easy to read. The \href{http://www.jython.org/}{Jython} project has migrated Python to the Java platform. Support for popular native libraries such as NumPy and SciPy are underway in a related project \href{http://www.jyni.org/}{JyNI} (the Jython Native Interface).

\subsection{Python and language details}

Leading white space is significant in Python; it is used to delimit code blocks. Avoid syntax errors by left-aligning code in multi-line string snippets as shown in the examples that follow.

\subsection{Python and Java classes}

To create a Java object in Python, simply import the associated package and call the class constructor. The \href{https://wiki.python.org/jython/NewUsersGuide}{Jython User Guide} provides further details for using Java classes. This example generates a random number using the \code{java.util.Random} class.

\begin{verbatim}
# Create an object from the java.util.Random class.
engine %~% "
from java.util import Random
r = Random(10)
"

# Jython supports invoking Java methods.
engine$invokeMethod("r", "nextDouble")

## [1] 0.7304303
\end{verbatim}

Jython's \code{jarray} module is required to create native Java arrays. It has two methods: \code{array(sequence, type)} and \code{zeros(length, type)}. The \code{array} method copies a Python sequence to a Java array of the given type. The \code{zeros} method initializes a Java array of the requested type with zero or \code{null}.

\begin{verbatim}
# Use 'jarray.array' to copy a sequence to a Java array of the requested type.
engine %~% "
from jarray import *
myArray = array([3, 2, 1], 'i')
"
engine$myArray

## [1] 3 2 1

# Alternatively, use zeros to initialize an array with zeros or null. This
# example allocates an array and udpates the values with a loop.
engine %~% "
myArray = zeros(5, 'i')
for i in range(myArray.__len__()):
  myArray[i] = i
"
engine$myArray

## [1] 0 1 2 3 4
\end{verbatim}

\subsection{A simple Python HTTP server}

This code sample creates a basic HTTP server in Python that calls R to produce HTML content. The Python script listed here defines two classes. The \code{MyHandler} class processes HEAD and GET requests for the server. The \code{MyServer} class is used by the R script to start and stop the web server. The Python code is adapted from the \href{https://wiki.python.org/moin/BaseHttpServer}{Python Wiki}.

%///test this code with line-wrapping enabled
\begin{verbatim}
import time
import BaseHTTPServer

# HTTP request handler class
class MyHandler(BaseHTTPServer.BaseHTTPRequestHandler):
    def do_HEAD(s):
        s.send_response(200)
        s.send_header("Content-type", "text/html")
        s.end_headers()
    def do_GET(s):
        print time.asctime(), "Received request"
        s.send_response(200)
        s.send_header("Content-type", "text/html")
        s.end_headers()
        s.wfile.write("<html><head><title>R/Python HTTP Server</title></head>")
        html = R.eval('getHtmlTable()') # Get HTML table from R.
        s.wfile.write(html)
        s.wfile.write("</body></html>")

class MyServer:
    def __init__(self, host_name, port_number, timeout):
        self.host_name = host_name
        self.port_number = port_number
        server_class = BaseHTTPServer.HTTPServer
        self.httpd = server_class((self.host_name, self.port_number), MyHandler)
        self.httpd.timeout = timeout
        print time.asctime(),
            "Server Starts - %s:%s" % (self.host_name, self.port_number)
    def handle_request(self):
        # This method exists only for demonstration purposes. For a production
        # scenario, see 'SocketServer.serve_forever()'.
        self.httpd.handle_request()
    def close(self):
        self.httpd.server_close()
        print time.asctime(),
            "Server Stops - %s:%s" % (self.host_name, self.port_number)
\end{verbatim}

The R script sources the Python script above and starts the web server. It also defines \code{getHtmlTable}: a function that generates HTML content for the web server. For demonstration purposes, the R script shuts down the Python web server automatically after 60 seconds. Run the R script and point a web browser to \url{http://127.0.0.1:8080} to see the result.

\begin{verbatim}
library("xtable")
library("jsr223")

server.runtime = 60 # Seconds before the HTTP server shuts down.

# Format the iris data set as an HTML table. This function will be called from
# the Python web server in response to an HTTP GET request.
getHtmlTable <- function() {
  t <- xtable(iris, "Iris Data")
  html <- capture.output(print(t, type = "html", caption.placement = "top"))
  paste0(html, collapse = "\n")
}

# Start the python engine.
engine <- ScriptEngine$new(
  engine.name = "python",
  class.path = "~/my-path/jython-standalone.jar"
)

# Source the Python script.
engine$source("./python-http-server.py", discard.return.value = TRUE)

runServer <- function() {
  # Automatically shut down server when this function exits.
  on.exit(
    {
      engine$invokeMethod("server", "close")
      engine$terminate()
    }
  )

  # Create an instance of Python 'MyServer' class which starts the server at the
  # specified port with the given request timeout in seconds. A timeout would
  # not be used in a production scenario.
  engine %@% "server = MyServer('localhost', 8080, 2)"

  # Handle requests for 'server.runtime' seconds before shutting down. The
  # 'handle_request' method waits for the timeout specified in the 'MyServer'
  # constructor before returning to the event loop to allow interruptions. In a
  # production scenario, the R side would not be involved in monitoring
  # requests. See Python's 'SocketServer.serve_forever()' for more information.
  started <- as.numeric(Sys.time())
  while(as.numeric(Sys.time()) - started < server.runtime)
    engine$invokeMethod("server", "handle_request")
}

runServer()
\end{verbatim}

\section{R with Ruby}

The \href{https://www.ruby-lang.org}{Ruby} programming language is a general-purpose, object-oriented programming language invented by Yukihiro Matsumoto. According to Matsumoto, he designed the language to “help every programmer in the world to be productive, and to enjoy programming, and to be happy” ///cite. \href{http://jruby.org/}{JRuby} is a Java implementation of the Ruby language. JRuby is compatible with the popular web application framework \href{http://rubyonrails.org/}{Ruby on Rails}. %///cite https://www.youtube.com/watch?v=oEkJvvGEtB4

\subsection{Ruby language details}

Global variables in Ruby script must be prefixed with a dollar sign. Hence, if we create a variable \code{myValue} using a \pkg{jsr223} assignment (e.g., \code{engine\$myValue <-\, 10}), it is accessed in Ruby script as \code{\$myValue}. Do not use the dollar sign prefix when accessing global variables via \pkg{jsr223} methods (e.g., \code{engine\$get("myValue")}).

We have observed a bug in JRuby's exception handling: when JRuby encounters an error, the engine may continue to throw errors erroneously in subsequent evaluation requests. If this happens, restart the script engine.

\subsection{Ruby and Java classes}

JRuby implements several methods to access Java classes in Ruby syntax. For a comprehensive guide, see \href{https://github.com/jruby/jruby/wiki/CallingJavaFromJRuby}{Calling Java from JRuby}. We demonstrate the most intuitive syntax using the multivariate normal random sampler.

\begin{verbatim}
library("jsr223")

# Include both the JRuby script engine and the Apache Commons Mathematics
# libraries in the class path. Specify the paths seperately in a character
# vector.
engine <- ScriptEngine$new(
  engine.name = "ruby",
  class.path = c(
    "~/my-path/jruby-complete.jar",
    "../commons-math3-3.6.1.jar"
  )
)

# Define the means and covariance matrix that will be used to create the
# bivariate normal distribution.
engine$means <- c(0, 2)
engine$covariances <- diag(1, nrow = 2)

# Import the class and create a new object from the class.
engine %@% "
java_import org.apache.commons.math3.distribution.MultivariateNormalDistribution
$mvn = MultivariateNormalDistribution.new($means, $covariances)
"

# This line would throw an error. JRuby supports 'invokeMethod' for
# native Ruby objects, but not for Java objects.
#
## engine$invokeMethod("mvn", "sample")

# Instead, use script...
engine %~% "$mvn.sample()"

## [1] 0.3279374 0.8652296

# ...or wrap the method in a function.
engine %@% "
def sample()
  return $mvn.sample()
end
"
engine$invokeFunction("sample")

## [1] 0.2527757 1.1942332

# Take three samples.
replicate(3, engine$invokeFunction("sample"))

##           [,1]      [,2]      [,3]
## [1,] 0.9924368 -1.295875 0.2025815
## [2,] 2.5145855  2.128243 1.1666272

engine$terminate()
\end{verbatim}

\hypertarget{usingrubygems}{\subsection{Using Ruby gems}}

Ruby libraries and programs are distributed in a standardized package format called a \textit{gem}. We demonstrate using gems in \pkg{jsr223} with Ty Rauber's \href{https://github.com/tyrauber/stock_quote}{\code{stock\_quote}} gem ///cite?.

A full installation of JRuby is required to use gems. Install JRuby and the \code{stock\_quote} gem using the instructions found in \href{https://github.com/jruby/jruby/wiki/GettingStarted}{Getting Started with JRuby}.%///cite?

To access \code{stock\_quote} with \pkg{jsr223}, the paths to the gem and its dependencies must be added to the \code{ScriptEngine\$new} class path. These paths can be discovered using the JRuby REPL, \code{jirb}, as shown here.

\begin{verbatim}
me@ubuntu:~$ jirb
irb(main):001:0> require 'stock_quote'
=> true
irb(main):002:0> puts $LOAD_PATH
~/jruby-9.1.5.0/lib/ruby/gems/shared/gems/unf-0.1.4-java/lib
~/jruby-9.1.5.0/lib/ruby/gems/shared/gems/domain_name-0.5.20160826/lib
~/jruby-9.1.5.0/lib/ruby/gems/shared/gems/http-cookie-1.0.3/lib
~/jruby-9.1.5.0/lib/ruby/gems/shared/gems/mime-types-2.99.3/lib
~/jruby-9.1.5.0/lib/ruby/gems/shared/gems/netrc-0.11.0/lib
~/jruby-9.1.5.0/lib/ruby/gems/shared/gems/rest-client-1.8.0/lib
~/jruby-9.1.5.0/lib/ruby/gems/shared/gems/stock_quote-1.2.3/lib
~/jruby-9.1.5.0/lib/ruby/2.3/site_ruby
~/jruby-9.1.5.0/lib/ruby/stdlib
=> nil
irb(main):003:0> exit
\end{verbatim}

Supply the resulting paths to the \code{class.path} parameter of the \pkg{jsr223} \code{ScriptEngine\$new} method. In our experience, the \samp{site\_ruby} path did not exist. If \code{ScriptEngine\$new} throws an error indicating a path does not exist, simply exclude it from the class path.

The code sample below downloads closing price history for two stocks and displays them in a combined time series plot (Figure \ref{fig:stocks}). The R packages \CRANpkg{jsonlite} ///cite and \CRANpkg{ggplot2} ///cite are required. %///Maybe do without jsonlite///

\begin{figure}[h]
\centering
\includegraphics[width=1\linewidth]{graphics/stocks}
\caption{Plot of closing stock prices produced by data obtained using Ty Rauber's \code{stock\_quote} Ruby gem.}
\label{fig:stocks}
\end{figure}

\begin{verbatim}
library("jsr223")
library("jsonlite")
library("ggplot2")

# In addition to the script engine JAR, include all of the required gem paths in
# the class path. In this case, we use 'jruby.jar' from the full installation
# instead of the standalone script engine JAR file.
#
# The gem paths were obtained by running the JRuby REPL 'jirb' in the terminal
# and executing the following two commands:
#
# require 'stock_quote'
# puts $LOAD_PATH

class.path <- "
~/jruby-9.1.5.0/lib/jruby.jar
~/jruby-9.1.5.0/lib/ruby/gems/shared/gems/unf-0.1.4-java/lib
~/jruby-9.1.5.0/lib/ruby/gems/shared/gems/domain_name-0.5.20160826/lib
~/jruby-9.1.5.0/lib/ruby/gems/shared/gems/http-cookie-1.0.3/lib
~/jruby-9.1.5.0/lib/ruby/gems/shared/gems/mime-types-2.99.3/lib
~/jruby-9.1.5.0/lib/ruby/gems/shared/gems/netrc-0.11.0/lib
~/jruby-9.1.5.0/lib/ruby/gems/shared/gems/rest-client-1.8.0/lib
~/jruby-9.1.5.0/lib/ruby/gems/shared/gems/stock_quote-1.2.3/lib
~/jruby-9.1.5.0/lib/ruby/stdlib
"
class.path <- unlist(strsplit(class.path, "\n", fixed = TRUE))

engine <- ScriptEngine$new(
  engine.name = "jruby",
  class.path = class.path
)

# Import the required Ruby libraries.
engine %@% "
require 'date'
require 'stock_quote'
"

# Print some real-time stock data to the console.
engine %@% "
$stock = StockQuote::Stock.quote('AEPGX')
puts $stock.name, $stock.change
"

# Ruby function to retrieve a year of closing stock prices for a given symbol in
# JSON format.
engine %@% "
def getStockYear(symbol)
  end_date = Date.today
  start_date = end_date.prev_year
  h = StockQuote::Stock.history(
    symbol,
    start_date = start_date,
    end_date = end_date,
    select = 'Close, Date',
    format = 'json'
  )
  return JSON.generate(h.fetch('quote'))
end
"

# R function wrapper to convert the Ruby JSON result to a data frame.
getStockYear <- function(symbol) {
  json <- engine$invokeFunction("getStockYear", symbol)
  df <- jsonlite::fromJSON(json)
  df$Close <- as.numeric(df$Close)
  df$Date <- as.POSIXct(df$Date)
  df
}

# Get closing values as data frames for two stocks.
aepgx <- getStockYear("AEPGX")
agthx <- getStockYear("AGTHX")

# Graph the closing values via 'ggplot2'.
ggplot() +
  geom_line(data = aepgx, aes(x = Date, y = Close, color = "AEPGX")) +
  geom_line(data = agthx, aes(x = Date, y = Close, color = "AGTHX")) +
  ggtitle("Closing Stock Prices: AEPGX, AGTHX") +
  theme(legend.position = c(0.1, 0.9), legend.title = element_blank()) +
  labs(x = "Date", y = "Closing Price")

engine$terminate()
\end{verbatim}

\section{R with Kotlin}

\href{https://kotlinlang.org/}{Kotlin} is a statically typed programming language that supports both functional and object-oriented programming paradigms. Kotlin is concise and pragmatic; in many cases, it requires less code than Java to accomplish the same task. For example, class definitions are particularly succinct. Kotlin version 1.0 was released in 2016 \citep{kotlin-release} making it the newest of the \pkg{jsr223}-supported languages. 

Kotlin's implementation of the Java Scripting API is progressing quickly though not complete. We will not list the deficiencies here as they will probably be resolved soon. If you experience a problem, refer to the \href{https://github.com/floidgilbert/jsr223/issues}{\pkg{jsr223} issue tracker} to review problems and workarounds. Select the ``Kotlin issues'' label and include both open and closed issues in your search.

\hypertarget{kotlinlanguagedetails}{\subsection{Kotlin language details}}

The Kotlin script engine handles bindings using a global map object instead of global variables. The best way to illustrate this behavior is by example. The following code creates and retrieves a binding \code{myValue} as you would expect.

\begin{verbatim}
engine$myValue <- 4
engine$myValue

## [1] 4
\end{verbatim}
However, \code{myValue} will not be available as a global variable in Kotlin script environment. Instead, it must be accessed and updated via the \code{jsr223Bindings} object as follows.

\begin{verbatim}
engine %@% 'jsr223Bindings["myValue"] = 3.1'
engine %~% 'jsr223Bindings["myValue"]'

## [1] 3.1
\end{verbatim}

Kotlin documentation demonstrates managing bindings through an object named \code{bindings}. However, the \code{bindings} object is read-only. This is a reported bug which may be fixed soon. Until then, the accepted workaround is to use \code{jsr223Bindings}.

In \hyperlink{callbacks}{Callbacks}, we explain how a global \code{R} object is added to the script engine environment to enable callbacks into the R environment. This \code{R} object is necessarily present in \code{jsr223Bindings}, but we do not recommend accessing it from that structure. Instead, use the global \code{R} variable as demonstrated in the code here.

\begin{verbatim}
# jsr223 automatically creates a variable R in the global scope of the Kotlin
# environment to facilitate callbacks.
engine %@% 'R.set("c", 4)'

# The R object in `jsr223Bindings` is inconvenient to use because it must be
# cast to an explicit type.
engine %@% '(jsr223Bindings["R"] as org.fgilbert.jsr223.RClient).set("c", 3)'
\end{verbatim}

\subsection{Kotlin and Java classes}

Kotlin is designed to be interoperable with Java. This simple example uses the \href{http://commons.apache.org/proper/commons-math/}{Apache Commons Mathematics Library} to sample from a bivariate normal distribution.

\begin{verbatim}
library("jsr223")

# Change this path to the installation directory of the Kotlin compiler.
kotlin.directory <- Sys.getenv("KOTLIN_HOME")

# Include both the Kotlin script engine jars and the Apache Commons Mathematics
# libraries in the class path.
engine <- ScriptEngine$new(
  engine.name = "kotlin"
  , class.path = c(
    getKotlinScriptEngineJars(kotlin.directory),
    "../commons-math3-3.6.1.jar"
  )
)

# Define the means and covariance matrix that will be used to create the
# bivariate normal distribution.
engine$means <- c(0, 2)
engine$covariances <- diag(1, nrow = 2)

# Import the package member and instantiate a new class.
engine %@% '
import org.apache.commons.math3.distribution.MultivariateNormalDistribution
val mvn = MultivariateNormalDistribution(
  jsr223Bindings["means"] as DoubleArray,
  jsr223Bindings["covariances"] as Array<DoubleArray>
)
'

# This line is a workaround for a Kotlin bug involving `invokeMethod`.
# https://github.com/floidgilbert/jsr223/issues/1
engine %@% 'jsr223Bindings["mvn"] = mvn'

# Take a multivariate sample.
engine$invokeMethod("mvn", "sample")

## [1] -2.286145  2.016230

# Take three samples.
replicate(3, engine$invokeMethod("mvn", "sample"))

##           [,1]      [,2]      [,3]
## [1,] 0.9924368 -1.295875 0.2025815
## [2,] 2.5145855  2.128243 1.1666272

# Terminate the script engine.
engine$terminate()
\end{verbatim}

%///\section{Package version history}

%///\section{Document version history}

\section{Conclusion}

%///people can use tools they know. Java is very accessible.
///

\clearpage

\nocite{*}
\bibliography{gilbert-dahl}

\address{Floid R. Gilbert\\
    Department of Statistics\\
    Brigham Young University\\
    Provo, UT 84602\\
    USA\\}
\email{floid.r.gilbert@gmail.com}

\address{David B. Dahl\\
    Department of Statistics\\
    Brigham Young University\\
    Provo, UT 84602\\
    USA\\}
\email{dahl@stat.byu.edu}

